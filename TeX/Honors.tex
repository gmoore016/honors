\documentclass[12pt]{article}

%Header + title info
\title{\vfill Investigating the Effects of Student Debt on Career Outcomes:\\ An Empirical Approach \vfill}
\author{An Honors Paper for the Department of Economics \\ \\ By Gideon Slocum Moore}
\date{}

%Bibliography info
\usepackage[style = authoryear, backend = bibtex]{biblatex}
\addbibresource{Sources.bib}

%Graphics info
\usepackage{graphicx}

%Misc packages
\usepackage{microtype}
\usepackage[includeheadfoot]{geometry}
\usepackage{booktabs}
\usepackage{blindtext}
\usepackage{amsmath}
\usepackage{amsfonts}
\usepackage{setspace} %For line spacing
\usepackage{xcolor} %For commenting visibly
\usepackage{appendix}
\usepackage{tabularx} %For tabout
\usepackage{longtable}

%Header (MUST BE LOADED AFTER GEOMETRY)
\usepackage{fancyhdr}
\pagestyle{fancy}
\lhead{Honors Thesis}
%\chead{Gideon Moore}
\rhead{Gideon Moore}

\usepackage{subcaption} %For subfigures

%Sets margins of paper
%Recommended between 1.25 and 1.5 inches
%Need etra quarter inch on left side due to binding
%Guidelines found here:
%https://bowdoin.libguides.com/c.php?g=271875&p=1812995
\geometry{margin=1in, lmargin=1.5in}


%Directory for regression results
\newcommand{\regs}{../Analysis/Regressions/Output/}


\begin{document}
	{\setstretch{1} \maketitle}
	\thispagestyle{empty}
	\vfill
	
	\setstretch{2}
	

	\begin{center}
		Bowdoin College, 2019

		\copyright \space 2019 Gideon Slocum Moore
	\end{center}

	\vfill

	\pagebreak
	
	\setcounter{page}{1}
	
	\section*{Abstract}
	
	High student debt has been hypothesized to affect career choice, causing students to desire more stable, higher paying jobs. To test this hypothesis, I rely on plausibly exogenous variation in debt due to a federal policy shift. In the summer of 2007, the Higher Education Reconciliation Act (or HERA) expanded the cap for federally subsidized student loans. I examine how variation in debt affects career choice and eventual salary of students using data from the National Longitudinal Survey of Youth 1979 Child and Young Adult Cohort of students who were of college age during the implementation of the policy. I find that student debt has no impact on salary two years after graduation; however, it does seem to shift students' career choices, leading some to avoid careers in public service industries such as teaching and social work.
	
	\pagebreak
	
	\section{Introduction}
	
	The returns to a college degree are large; in 2014 the expected return to a college degree was more than twice that of standard investment vehicles such as stocks. That return was rising partially due to increasing wages of the skilled, but moreso due to falling wages for those lacking a degree \parencite{abel2014}. Partially in response to this high return, from 2000 to 2015 total undergraduate enrollment in the U.S. rose by 30 percent from 13.2 to 17 million students \parencite{mcfarland2017}. These returns only continue to rise despite rising costs of tuition; from 1995 to 2015, the real cost of a college degree rose by 65 percent \parencite{doe2016}. Because tuition continues to rise, many American families expect to take out loans in order to afford college for themselves or their children. This has caused student debt in the U.S. to balloon, tripling in real terms from \$340 billion to more than \$1.3 trillion between 2000 and 2015 \parencite{feiveson2018}. Today, student loans are the second largest source of household debt, falling only behind mortgages \parencite{dynarski2015}. 
	
	Traditional models of investment would suggest the choice of financing should not change consumer behavior \parencite{modigliani1958}--for a given student, in the absence of credit constraints, whether they finance their education with debt or existing capital would not change their consumption decisions. However, empirically this is not the case. A survey by Nellie Mae suggests that one in six loan recipients have changed their career plans due to their debt obligation, though how those plans changed was not specified in the survey \parencite{baum2003}. This survey contained only those who had not yet defaulted on their debt; as defaulters were heavily affected by debt, the true number is likely greater. \textcite{minicozzi2005} finds that men who take on more undergraduate debt take jobs with higher initial wages but lower year-over-year wage growth, again suggesting a real impact of what should be a purely accounting-based change. 
	
	A debt subsidy could theoretically create an income effect, but it is likely small. Even an extremely generous program promising zero interest on student debt would only increase students' income by around \$50 per year per thousand dollars borrowed--a fraction of the expected lifetime earnings of a college attendee. Thus, it is unlikely the income effect from this subsidy would be able to explain changes in life decisions such as career path.
	
	An alternate explanation as to why debt would change students' behavior is that students may be ``debt averse,'' meaning that the act of holding debt has a negative impact on students' utility independent of any impact that debt may have on later consumption. This would cause students to pay off debt more quickly than would be efficient, ``consuming'' through paying off debt ahead of schedule. Students who are minorities, low-income, or first generation are especially susceptible to this phenomenon \parencite{burdman2005, field2009, callender2005}. 
	
	One frequent explanation for this phenomenon is risk aversion. Given student debt is uniquely difficult to disperse, the consequences for default can be catastrophic. Thus, as students take on more debt, they risk failing to pay back their loans. It is rational, then, to gravitate to higher paying, more stable careers which could bring a lower risk of insolvency. 
	
	The most relevant study on debt aversion is by \textcite{field2009}. In her experiment, New York University law students were randomly assigned either loan- or grant-based aid. The students assigned loans were told that their debt would be forgiven upon entry into a public service sector. Similarly, students given grants were told that, if they did \emph{not} enter the public service sector, they would be obligated to repay their grants. The two programs were designed to be financially identical--they differed only in the framing of whether the student was ``in debt.'' However, the author found that the students who were given the debt-based aid were significantly more likely to choose the high-wage, non-public-service position than their counterparts receiving grant-based aid. Given the aid packages are financially identical, the authors conclude this likely derives from some negative socio-psychological impact of carrying debt.
	
	In an ideal world, one could answer the question of how debt affects employment outcomes through regressing some measure of career choice on debt level. However, endogeneity issues make this solution infeasible. Given that students choose to take debt based on expectations of their future earnings, reverse-causality is a major concern. Without random assignment or an instrument, we cannot accurately infer the direction of causality. 
	
	The focus of this paper is to determine whether increases in student debt lead students to change their fields of study and eventually career choices. While prior work in the literature has largely focused on policy changes within a single institution, I hope to create a more representative picture of the nation as a whole using data from the National Longitudinal Survey of Youth. In 2006, the United States raised the cap for Stafford Loans for freshmen and sophomores, creating a natural experiment wherein the students before the policy change were more limited in their access to debt than the students after the shift.
	
	I find that student debt has negligible impact on incomes two years after graduation. However, students with more student debt are significantly less likely to work in public service positions such as education and social work. These career choice findings are in line with \textcite{rothstein2011} who also find reduced engagement in public service sectors. Also in line with Rothstein and Rouse, I find that this impact is not present in other sectors; student debt appears to have no impact on students' choices to enter finance, STEM, or the humanities. This appears to be in line with a model I propose where given a greater credit constraint, students substitute away from careers with non-monetary benefits towards those with a greater focus on remuneration. If those in public service receive some sort of non-monetary benefit from serving in an altruistic career as suggested by \textcite{hanson1995}, or generate other positive externalities, this shift in preference could negatively impact society utility as those benefits are lost when students migrate away from these public service fields.
	
	\section{Literature Review}
	
	The current central paper of the literature is \textcite{rothstein2011}, which uses the introduction of a ``no loans'' policy at prestigious private college ``Anon U''--implied to be Princeton University--as a natural experiment on the enrolled students. The authors use a function of expected family contribution as an instrument for the impact of the policy, discovering that an additional \$10,000 of student debt leads students to reduce employment in the non-profit and education sectors by 5.2 percent and reduce employment in low-wage sectors generally by 5.7 percent. This is further corroborated by an expected \$2,000 bump in wages per \$10,000 of debt. The authors also suggest increased debt shifts students towards more ``employable'' majors over what they describe as ``consumption'' majors. The main shortfall of this paper which I hope to correct is that Princeton students are not representative of the general population -- as of 2017, nearly 60 percent of students at Princeton came from families in the top 10 percent of Americans by income \parencite{aisch2017}. Other papers have examined similar phenomena in more specialized and similarly wealthy fields, finding that increased debt leads both dentists and lawyers to accept higher-paying jobs in disproportionately private practices \parencite{nicholson2015, field2009}.
	
	One suggestion for this form of behavior is the previously mentioned ``debt aversion,'' where borrowers are less inclined to take and carry debt than a traditional model would indicate. \textcite{burdman2005} finds many non-income factors can influence whether students choose to borrow to attend school. One of the most influential factors she found was parental attitudes towards loans, suggesting that students whose parents were debt averse were more likely to be debt averse themselves. \textcite{callender2005} find that low-income and minority students are significantly less likely to borrow, regardless of their field of study. However, \textcite{eckel2007} find that students who are debt averse are not significantly less likely to have borrowed to attend school, suggesting that I can expect to see debt aversion in mysample. 
	
	One common thread among many of the above papers, however, is their relative neglect of low income students. \textcite{rothstein2011} is the paper most similar to mine in the literature; however, their exclusive focus on Princeton students make it difficult to extrapolate the results to most American college students. \textcite{field2009} examines a similar question to mine as well, but examines only law students at New York University, again failing to use a sample applicable to the majority of American students. Given that \textcite{callender2005} shows that low income students have significantly different attitudes towards debt than their wealthier peers for reasons such as debt aversion and parental attitudes towards debt, it is clear that low income students require special attention in questions of student loans and career choice. Thus, by using a nationally representative sample provided by the NLSY, I hope to provide novel insight into the relationship between debt and career choice among lower- and middle-class students.

	\section{Theoretical Model}
	
	Before beginning my empirical work, I develop a structural model in order to help interpret the results visible in the data. This model suggests that there exists a direct trade-off between financial stability and non-monetary job benefits caused by dependence on debt. Thus, as debt use increases, I expect students to shift away from careers with significant non-monetary benefits towards those which provide greater financial stability.

	I propose a model of career choice and loan repayment similar to that developed by \textcite{abraham2018}.\footnote{Their model, while also examining students' decisions between risky and safe careers, considers only monetary payoffs. Thus, my addition of a non-monetary payoff $C$ which is unavailable as collateral against the loan is a novel addition to their model. In addition, their model considers probability of default and loan size to be unrelated; however, in my model I claim the probability of default rises as the size of a loan grows, as suggested in \textcite{dynarski2015}, which lends the model additional richness when using it to examine the effect of bounded credit.} 
	Consider a two-stage consumption model. Individuals consume in period one entirely through borrowing against period two earnings, labeled as $B$. In period one agents choose to pursue a higher-paying career path $H$ or a lower-paying career path $L$. Consumption in period two is composed of two parts, less the quantity borrowed in period one. First, $C$, non-monetary utility from the chosen career path; consider the value of altruism for working in a non-profit or the value of expression for working as an artist. The second consumption parameter in time two is monetary salary, denoted $H$ and $L$ for the higher- and lower-paying careers respectively.
	
	Debtors are guaranteed to pay back their loan on career path $H$ due to their higher income. When taking career $L$, repayment is no longer guaranteed; instead, students repay their loan with probability $p$. If students default (with probability $1 - p$), their monetary compensation for the period is reduced to zero, leaving only non-monetary compensation $C$ which is unique to choice $L$. I assume that \emph{ceteris paribus} larger loans are more difficult to repay; thus, $p$ is a function decreasing in $B$. Finally, there is some discount rate $r$ for the money borrowed in period one. This $r$ will favor the present, as I am examining interest-free loans for the students. This value may be interpreted as the market interest rate, as there is nothing in my model preventing students from borrowing their loans interest-free and re-investing them into the market. 
	
	The above definitions yield the following two utility functions for a risk-neutral consumer: 
	\begin{align}
	\mathbb{E}\left[U_H(B)\right] &= B + H - (1 - r)B \label{highu}\\
	\mathbb{E}\left[U_L(B)\right] &= B + C + p(B) \times L - (1 - r)B \label{lowu}
	\end{align}
	
	Note that students choosing career $L$ over career $H$ suggests that equation \ref{highu} is greater than equation \ref{lowu}. Some algebra yields the following equivalence relationship: 
	\begin{equation}
	\mathbb{E}\left[U_L(B)\right] - \mathbb{E}\left[U_H(B)\right] > 0 \iff p(B) > \frac{H - C}{L} \label{choicecon}
	\end{equation} 
	Thus, there are five factors that could shift students from career $H$ to career $L$:
	\begin{itemize}
		\singlespacing
		\item A decrease in the high paying career's salary $H$
		\item An increase in non-monetary utility for the low paying career $C$
		\item An increase in the low paying career's salary $L$
		\item An increase in quantity borrowed $B$, as $p'(B) < 0$
		\item An increase in the probability of repayment $p$ for the given borrowing level $B$
	\end{itemize}

	The above model is deterministic for any given student; given a fixed $H, C, L, p,$ and $B$ they will always choose the same career. Moreover, for simplicity, I allow $H, L$, and $p$ to be constant across all students, representing access to similar credit and labor markets. However, suppose there exists idiosyncratic variation in $C$ across students; student $i$ may derive great utility from career $L$, while student $j$ may not care for $L$ at all.

	Note that for each student there exists some $B^*$ such that they are indifferent between $H$ and $L$. Suppose there exists some global upper bound on borrowing $\bar{B}$ which is binding for both those on career paths $H$ and $L$. When $\bar{B}$ is increased as in Figure \ref{struc} from $B1$ to $B2$, this will cause some--but not all--students to shift career from $L$ to $H$ as $\bar{B}$ may shift to be greater than $B^*$ for some subset of students. This change in career choice is intuitive; given larger debts are more difficult to repay, students with more debt are more likely to choose the career guaranteeing solvency. Given students before and after this change are comparable, I could treat this shift as a natural experiment, providing each student after the shift with an additional $\Delta\bar{B}$ debt burden. 
	
		
	\begin{figure}
		\centering
		\caption{Illustration of Structural Model}
		\label{struc}
		\includegraphics[width = 0.8\textwidth]{Images/ModelSketch}
	\end{figure}

	This model provides a concrete prediction regarding agents' behavior under a positive credit shock. In the absence of other factors, students who receive access to an increased line of credit should pivot \emph{away} from occupations with non-pecuniary benefits (those with higher $C$) towards those which are more lucrative. Depending on the context for $C$, this could have important consequences for a social planner. For example, \textcite{benshem1991} suggest that many in ``helping professions'' such as education, rehabilitation, and social work, employees choose their careers due to valuing altruism over other choice factors. Similarly, \textcite{hanson1995} found that social workers frequently ranked ``contributing to society'' and ``effecting social change'' above private benefits such as ``Job Security'' or ``Good Working Conditions.'' In this case, altruism is the $C$ which drives students to choose career $L$; a social planner, then, would wish to preserve this $C$ in order to maintain the positive externalities which would be lost if those students instead chose career path $H$.

	\section{HERA and Subsidized Student Loans}
	
	Turning now to my empirical methods, I devise a model predicting outcomes such as salary and career choice as a function of debt as well as other background characteristics. While I would like to regress career outcomes directly on student loans, the quantity of loans is an endogenous variable. For example, if students from lower income backgrounds take out more loans, but also have lower salaries for reasons unrelated to their educational spending such as weaker professional networks, this would cause loans to be negatively correlated with salary despite the loans not having caused the salary reduction. On the other hand, if students from higher income backgrounds have access to more credit, they may take out more loans but also have higher salaries for unrelated reasons, suggesting a positive relationship between debt and salary. Among other mechanisms, these two examples demonstrate the endogeneity present in the relationship between loans and career outcomes. To correct for this endogeneity, I perform a two-stage least squares regression using a 2006 shift in federal loan policy to pseudo-randomly influence students' debt, mitigating the endogeneity problem. Assuming that students before and after the policy are comparable (an assumption I support in section 6), the students after the policy have received a roughly 20 percent increase in subsidized loans from the federal government. Thus, by comparing these students to those before the policy, I can parse out the causal effects of these additional loans.
	
	I obtain exogenous variation in debt using the identification strategy of \textcite{lucca2018} who examine how increased Federal subsidized credit lines impact university tuition. In 2006, the Higher Education Reconciliation Act (HERA) increased subsidized borrowing caps for the first time in fourteen years. Freshmen received boosts from \$2,625 to \$3,500 per year, while sophomores received a greater boost from \$3,500 to \$4,500 per year. Caps for older students remained static at \$5,500. While HERA also increased the availability of unsubsidized credit, Lucca and his coauthors make a compelling argument that the increase in unsubsidized loan caps had a middling effect at best on uptake for unsubsidized loans. Students who increase unsubsidized borrowing under the program shift would have to meet two criteria: first, the students would have to be well off enough to not qualify for a wholly subsidized program, and second, the students still chose to take advantage of their whole program allowance. The authors find that fewer than 1 percent of the students in their sample met both of these criteria \parencite[p. 434]{lucca2018}.
	
	The act became effective during early summer 2007, meaning students had plenty of time to take advantage of the new rates before the fall semester. Thus, there is a clean break between the students with and without the lower credit limit moving from academic year 2006-07 and 2007-08.
	
	The authors also provide evidence that the subsidized loan cap had previously been binding. Figure \ref{luc} shows their histograms of student-level loan quantities before and after the HERA shift from the New York Fed/Equifax Consumer Credit Panel. Figure \ref{luc06} has a clear mode of \$2,625, the exact subsidized cap for freshmen pre-HERA, and a secondary mode of \$3,500, the subsidized cap for sophomores. Compare this with Figure \ref{luc07}, which has modes at \$3,500 and \$4,500, the post-HERA caps for freshmen and sophomores, respectively. Further note that both graphs contain clusters at \$5,500, the constant cap for upperclass students, again suggesting the binding nature of the subsidized loan cap. This is relevant to my analysis as it suggests that raising the borrowing cap will induce students to borrow more. Given that the cap on subsidized loans is binding, I argue that students would borrow more if given the opportunity. This is especially true when there is little dispersion below the cap as seen in Lucca et al.'s histograms in Figure \ref{luc}; nearly all students appear to borrow as much as they are offered, suggesting many students would change their behavior given a greater borrowing ceiling.
	
	\begin{figure}
	\centering
	\caption{Student Loan Frequencies From NY Fed CCP/Equifax Panel as Presented in \textcite{lucca2018}.}
	\label{luc}
	\begin{subfigure}{0.49\textwidth}
		\centering
		\caption{Student Loan Frequencies 2006-2007}
		\label{luc06}
		\includegraphics[width = \linewidth]{Images/Lucca6aBW.png}
	\end{subfigure} 
	\begin{subfigure}{0.49\textwidth}
		\centering
		\caption{Student Loan Frequencies 2007-2008}
		\label{luc07}
		\includegraphics[width = \linewidth]{Images/Lucca6bBW.png}
	\end{subfigure}
	\end{figure}

	
	Given that students before and after the policy shift are comparable, this exogenous shift in credit line allows the separation of the effects of need and debt within the data. A student with a fixed level of need in 2009 would have roughly \$2,000 more debt than a student with a comparable level of need in 2007. Thus, we can examine the impact of the debt separately from the student's level of need. A scatter plot demonstrating the variance between need and loans is available in Figure \ref{needvloans}
	
	\begin{figure}
		\centering
		\caption{Student Loans vs. Need, Excluding Major Outliers}
		\label{needvloans}
		\includegraphics{../Analysis/Charts/Output/LoansvsNeed}
	\end{figure}
	
	
	\section{Data}
	
	For my paper I use data from the National Longitudinal Survey of Youth '79 Children and Young Adults, or NLSY79 CYA \parencite{bls2018}. A follow-up to the original National Longitudinal Survey of Youth '79, the CYA surveys the original respondents' children every other year from birth until the present day. Thus, I have respondents across the entire age spectrum during the implementation of the policy, allowing us to observe time-varying trends in debt acquisition and career choice. 
	
	The benefits of the CYA data over the Anon U data from \textcite{rothstein2011} are two. First, the NLSY is designed to focus on lower income households rather than Anon U's much wealthier sample. As visible in Figure \ref{incDist}, the CYA sample is weighted towards middle- and lower-income households with a median income of \$70,000 in year 2000 dollars. Compare this with Princeton (believed to be Anon U), which has a median family income of \$186,100 \parencite{aisch2017}--roughly \$126,000 in year 2000 dollars. The average treated student in my sample (that is, the average student with need greater than or equal to the starting policy cap) had a family income of roughly \$29,600 in year 2000 dollars, compared with Anon U whose average treated student had a family income of \$59,900 in year 2000 dollars. The students in my sample are clearly from a much poorer economic background than those at Anon U, a group which has historically been underrepresented in these types of studies. Second, the students in this survey attend a much broader range of colleges than those in the Rothstein data set. As the students who are accepted into prestigious universities such as Anon U are by definition selected, they \emph{cannot} be representative of the general American college-bound population. 
	
	\begin{figure}
		\centering
		\caption{Distribution of Parental Income in NLSY 1979 CYA Sample}
		\label{incDist}
		\includegraphics[width = 0.9\textwidth]{../Analysis/Charts/Output/parIncDist}
	\end{figure}
	
	One drawback of the CYA data set is that, as a survey, many of the figures are less precise than those in other sets. For example, when asked about their level of debt, the vast majority of responses are rounded numbers. While useful for qualitative analysis, this makes it difficult to assess data granularly as done by \textcite{lucca2018} in Figure \ref{luc}. Fortunately, because they have a nationally representative dataset, there is no reason to believe their results would differ with my sample. Indeed, as my sample is of lower income relative to the general population, I would argue that subjects are bound more strongly by the policy as for many students federal loans may be their only option for college financing. In addition, as visible in Figure \ref{loanDist}, 85 percent of students received \$10,000 or less in debt, suggesting an additional \$1,000 would be meaningful relative to their overall quantity. Further, because observations are survey results rather than financial data, I have a relatively smaller sample size as data is only collected every other year, and only from a subset of the total population.
	

	\begin{figure}
		\centering
		\caption{Distribution of Loan Size in NLSY 1979 CYA Sample}
		\label{loanDist}
		\includegraphics[width = 0.9\textwidth]{../Analysis/Charts/Output/loanDist}
	\end{figure}

	
	Because of the connection between the CYA and original NLSY79, I can import very rich background information for each of my participants. Specifically, for each student I can identify parental income for the year in question among other background traits inherited from parents. I adjust all of these variables for inflation using the Chained Consumer Price Index for all Urban Consumers \parencite{bls2019}. However, the drawback of being linked to the prior survey is that many of my observations are siblings, which would imply that they are not independently distributed if they share unobservable characteristics. To combat this issue, I will be clustering all standard errors based on the students' mothers. 
	
	Unfortunately, while the NLSY does contain information about each student's debt level, it does not break down these loans by type. Thus, from the given data I cannot tell how much debt is subsidized through the Stafford program and how much is unsubsidized. I overcome this hurdle by using other background characteristics of the students to impute how much subsidized debt they were offered based on their expected family contribution (or EFC) from the 2007 Free Application for Federal Student Aid (or FAFSA) \parencite{doe2007}. If students choose to take subsidized debt over unsubsidized debt, then I can assume any quantity of debt below their calculated need must be subsidized, as the government offers subsidized debt to students up to their need or the program cap, whichever is lower. Therefore, by calculated the EFC, I can impute how much subsidized debt students could access.
	
	The FAFSA bases your household's expected family contribution, and therefore your debt allotment, on mother's age, number of siblings, household income, and household assets. While most of these are easily accessible in the NLSY dataset, respondents are only surveyed about their assets every four years, and only a subset of the survey participants are surveyed. I take two primary steps to rectify this. First, for those subjects who were surveyed about their assets, I extend the last known value into the missing year. For example, if asset data is missing in 2006, I use the subject's survey value for 2004. Second, for those parents with no existing asset data, I infer assets by performing the following regression on those whose asset data I do possess: 
	\begin{equation}
	\begin{aligned}
	\mbox{Assets} = \beta_0 &+ \beta_1 \mbox{Income} + \beta_2 \mbox{Income}^2 + \beta_3 \mbox{Age} \\
	&+ \beta_4 \mbox{Children} + \beta_5 \mbox{Year} + \vec{\beta} \vec{\mbox{Demographics}} + \epsilon
	\end{aligned}
	\end{equation}
	Since the subjects of the subsurvey were also chosen to be representative of the population, I feel comfortable that I am not extrapolating this data too far. I choose to let income be quadratic to reflect increasing marginal propensity to save among the wealthy. This regression yields the results visible in Table \ref{assetTab}. The fit and model appear to be reasonable, so I use this relationship to impute asset values.
	
	\begin{table}
		\centering
		\caption{Estimation of parental assets}
		\input{../Analysis/NeedCalc/Output/assethat.tex}
		\label{assetTab}
	\end{table}

	To ensure the policy shift is not being confounded with a shifting student body, I must ensure that the composition of students in my sample is constant before and after the policy change in 2007. Fortunately, my samples before and after the policy change appear to be comparable across a variety of dimensions. The distributions of students before and after the policy change are visible by race in Table \ref{racecomp}, sex in Table \ref{sexcomp}, and region in Table \ref{regioncomp}. Based on the $\chi^2$ values of these distributions, I can be relatively confident that the composition of the student body before and after the treatment is comparable across these factors. In addition, I conduct a two-sample t-test comparing (inflation-adjusted) parental incomes across the groups. Before 2008, there are 271 observations with a mean parental income of \$67,945. From 2008 on there are 484 observations with a mean parental income of \$74,806. Testing whether the means of these two groups are the same yields a $t$-statistic of 1.40 ($p = 0.16$). Thus, I fail to reject the hypothesis that real parental income in the sample remained constant before and after the policy shift. These results are consistent with the literature on the topic; \textcite{dynarski2015} claims that despite the lowered cost of college from the interest subsidy, interest-based incentives are not ``salient.'' Because the subsidized interest is not visible at the ``moment of decision,'' students do not take it into account when making college enrollment decisions. 
	
{
	\begin{table}
		\centering
		\caption{Comparison of race before and after policy change}
		\resizebox{\textwidth}{!}{
			\begin{tabular}{lrrrrrrrr}
				\input{../Analysis/Tables/Output/raceFit.tex}			
			\end{tabular}
		}
		\label{racecomp}
	\end{table}

	\begin{table}
		\small
		\centering
		\caption{Comparison of sex before and after policy change}
		\begin{tabular}{lrrrrrr}
			\input{../Analysis/Tables/Output/sexFit.tex}
		\end{tabular}
		\label{sexcomp}
	\end{table}

	\begin{table}
		\centering
		\caption{Comparison of region before and after policy change}
		\resizebox{\textwidth}{!}{
			\begin{tabular}{lrrrrrrrrrr}
				\input{../Analysis/Tables/Output/regionFit.tex}
			\end{tabular}
		}
		\label{regioncomp}
	\end{table}
}
	
	\section{Statistical Model}
	
	My data set is a collection of college seniors surveyed on even years between 2000 and 2014. I model career outcome $Y$ according to the following equation: 
	\begin{equation}
	Y = \beta_0 + \beta_1 \mbox{debt}_{it} + \vec{\xi}^T \vec{\mbox{demographics}}_i + \epsilon_{it} \label{naiveeq}
	\end{equation} 
	In this model, $Y$ could take one of several values, such as major choice, career choice, or salary. 
	
	My first model will place major choice on the left hand side. I believe this to be the most novel outcome of interest, as few datasets have this level of resolution on their observations. Here I can place majors into several bins, and then run a probit regression on individual bins to determine the effect of the policy. I base my major groups off of those used in \textcite{rothstein2011}. Here a ``public service'' major is defined to be working in either education or social work; while Rothstein and Rouse also include work in non-profits, my data set does not include designations on employer type. However, given that they attribute their effects largely to the education sector, and the majority of their public service sector is involved in education, I argue my designation is comparable. ``Finance'' majors are defined to be students majoring in Economics, Business, or Home Economics, as the NLSY groups finance majors as a form of business major and several Finance programs are subdepartments of Economics. Again following Rothstein and Rouse's specification, ``lucrative'' majors are finance majors as well as particularly well-compensated STEM fields, specifically Engineering and Computer Science. Finally, ``humanities'' majors are defined to be Language, Anthropology, Philosophy, Theology, Art, and English.
	
	My second model will directly examine students' final incomes. This is the most direct approach to measuring the impact of debt on choices; however, it may ignore factors such as career risk which would not be visible in simple income data.
	
	The results of this naive model using several types of majors as dependent variables are visible in Table \ref{naive2} and the marginal effects are visible in Table \ref{naive2marg}. I show that under the naive specification, loans appear completely irrelevant to career choice in all fields.
	
	
	\begin{table}
		\centering
		\caption{Results of the naive regression for career choice}
		\resizebox{\textwidth}{!}{
			\input{\regs naive.tex}
		}
		\label{naive2}
	\end{table}

	\begin{table}
		\centering
		\caption{Marginal effects of the naive regression for career choice}
		\resizebox{\textwidth}{!}{
			\input{\regs naivemarg.tex}
		}
		\label{naive2marg}
	\end{table}
	
	As outlined in section 4, however, it is likely there exists endogeneity between loans and career outcomes; therefore, I perform a two-stage least squares regression using exposure to the HERA credit increase as my instrumental variable in order to pseudo-randomly assign debt to the students. 
	
	I assume that outside of my random assignment, debt levels are dependent on parental resources, time trends, and demographic characteristics such as sex, race, and region of residence. As the time trend for debt levels may not be linear, I use dummies for each year to represent average debt within my sample for a given year. Further, I anticipate students taking out loans each semester; thus, those interviewed in the fall of their senior year would have less debt than those in the spring, so I include a semester dummy as well.
	
	The ideal first-stage model, then, would be: 
	\begin{equation}
	\begin{aligned}
	debt_i = \delta_0 &+ \delta_1 treatment_i + \delta_2 parIncome_i + \delta_3 fallSemester_i \\
	&+ \vec{\delta}^T i.year_i + \vec{\gamma}^T \vec{demographics}_i + \mu_i \label{fseq}
	\end{aligned}
	\end{equation}
	
	There exist two primary problems with this approach, however. The first is that debt is inherently left-censored. While students who take loans display some positive value, those who choose not to borrow display zero when they instead potentially chose some ``negative'' through investing their assets rather than borrowing. 
	
	The solution to this issue is to perform the first stage regression using a Tobit model rather than traditional OLS, thus bounding predicted debt at zero. However, this creates problems of its own from using the fitted values directly as regressors (Hausman 1975 as cited in Angrist and Pischke 2009).	 The Tobit is nonlinear, and therefore it is no longer guaranteed that these fitted values are linearly uncorrelated with the structural residuals in equation \ref{naiveeq}. I use a method suggested by \textcite{angrist2009} to correct this issue. If my model is correct, then the values predicted in the above first stage model will strongly predict career outcomes in the second stage. Therefore, I can use the fitted values of debt from the first Tobit as instruments for actual levels of debt in the second stage, akin to a traditional Two Stage Least Squares model. By using a fitted linear model in the instrumenting stage, that is, the last stage, I guarantee that structural residuals will be uncorrelated with the fitted values and covariates in the usual way. 
	
	The second difficulty this model faces is that it is difficult to determine which students are ``treated,'' as not all students observed after the policy change received the additional credit. Since subsidized loans are targeted at low-income families, only students whose anticipated need is greater than the existing cap were subject to this increased credit line. 
	
	Anticipated need is equal to students' tuition payments minus their EFC. While I cannot observe students' true EFC, I can calculate an approximation using the FAFSA, as outlined in section 4. Then, using this approximation, I am able to generate a measure of need and so determine who is affected by the policy shift. I then generate a dummy equal to 1 if a student's need is greater than the old borrowing limit, \$4,500 in nominal terms. This would imply treatment by the policy given they attend school 2008 or later.
	
	As an initial test of the impact of the policy, I regress loans on students' need and whether they are at the cap using two different cross-sections of students, one before and one after the policy shift. If my instrument is valid, I would expect the coefficient on estimated need to be larger in the post-policy shift sample. That is, students after the policy shift are able to take out greater levels of debt based on their level of need given they are at the previous cap for demonstrated need due the increased credit limit. The results of this initial regression are visible in columns (1) and (2) respectively of Table \ref{firststage}. In this specification the coefficient for need above the cap appears more positive after the policy change than before, as expected. 
	
	In my more rigorous model, I perform a triple difference approach, interacting $atCap$ with students' need and policy shift. This allows me to use both students before the policy shift (who experience no policy impact) and students after the policy shift who are below the threshold (and thus have $atCap = 0$) as controls for my cohort of interest. Thus, my final model for the first stage is as follows: 
	\begin{equation}
	\begin{aligned}
	debt_i = \delta_0 &+ \delta_1 need + \delta_2 (need \geq 4500) + \delta_3 (need \geq 4500) \times need \\ 
	&+ \delta_4 policyImpact + \delta_5 (need \geq 4500) \times policyImpact \\
	&+ \delta_6 policyImpact \times need + \delta_7 (need \geq 4500) \times policyImpact \times need \\
	&+ \delta_8 fallSemester + \vec{\delta} i.year + \vec{\gamma} \vec{demographics}^T
	 \end{aligned}
	 \end{equation} 
	 
	 The results of this model are given in column (3) of Table \ref{firststage} with time and demographic effects omitted. An interpretation of each interaction coefficient is available in Table \ref{coefInterp}. The coefficients of interest are those involving the dummy for $need \geq 4500$, as those coefficients together indicate the impact of my instrument. I run a Wald test on these coefficients ($\delta_2, \delta_3, \delta_5,$ and $\delta_7$). The Wald test examines whether all coefficients tested are jointly equal to zero. I find an $F$ statistic of 2.40 and a $p$ value of $0.05$, suggesting that all coefficients together are not jointly zero. This, in conjunction with my regression results, makes me comfortable with my instrument's predictive power.
	 
	 %Results of my various first stages
	 \begin{table}
	 	\centering
	 	\caption{Results of first stage Tobit regression of loans on policy impact}	 
	 	\resizebox{\textwidth}{!}{	
		 	\input{\regs tripdif.tex}
		 }
	 	\label{firststage}
	 \end{table}
	
	%Interprets different interaction effects
	\begin{table}
		\centering
		\caption{Interaction Coefficient Interpretations}		
		\begin{tabular}{lp{8cm}}
			\toprule
			Coefficient & Interpretation\\
			\midrule
			$\delta_1$ & Impact of an additional dollar of need\\
			$\delta_2$ & Impact of having need greater than \$4,500 \\
			$\delta_3$ & Additional impact of each dollar of need over \$4,500 \\
			$\delta_4$ & Impact of an additional dollar of credit available for those below the cap (expected to be zero)\\
			$\delta_5$ & Additional impact of an additional dollar of credit for those at the cap, not considering degree of need\\
			$\delta_6$ & Impact of an additional dollar of credit available for those not at the cap (expected to be zero)\\
			$\delta_7$ & Additional impact of an additional dollar of credit availability for those at the cap proportional to the student's need \\
			\bottomrule
		\end{tabular}
	
		\label{coefInterp}
	\end{table}
	
	
	\section{Results}
	
	I begin by estimating the likelihood of several types of careers. This will be done in the form of a probit model: either a student chooses to work in the relevant field, or they do not. Due to the nonlinearity of the tobit model used in the first stage, I use the estimates of debt from that equation as an instrument in this model (called $\hat{loan}$). The form of the regression is as follows: 
	\begin{equation}
	\mathbb{P}\left(major_i = majType\right) = \Phi\left(\beta_0 + \beta_1 \hat{loan_i} + \beta_2 year + \vec{\beta}\vec{demographics}^T_i\right)
	\end{equation}
	
	The abbreviated results of these regressions are in Table \ref{majChoice} with the marginal effects visible in Table \ref{majChoicemarg}. Be aware that all regression tables in this section are shortened to the coefficients of interest; full results are available in the appendix. As consistent with Rothstein and Rouse's findings, students with more debt are less likely to choose a major in public service, in this case defined to be teaching or social work. These results are visible in regression (1). When converted to marginal effects, this suggests that every extra thousand dollars of debt a student receives lowers their probability of entering public service by roughly 5.5 percent. While this effect is significantly larger than that of Rothstein and Rouse, I would claim this is logical--as the students receiving these subsidized loans are significantly less wealthy than Princeton students, I would expect a smaller shift to be necessary to generate comparable results. Also consistent with Rothstein and Rouse's findings, people are \emph{not} significantly more likely to go into more lucrative fields such as business, economics, engineering, or computer science, as visible in regressions (2) and (3). 
	
	Also interesting is that I find needier students are more likely to work in public service as well. When converted to marginal effects, an extra thousand dollars of demonstrated need raises the probability of working in a public service position by 0.3 percent. However, need does not appear to have impacts on the likelihood of entering other fields.
	
	To illustrate the impact of these effects, I have calculated the probability that a given student in our sample will work in public service for various levels of need and student debt. These estimates are visible in Table \ref{probest}. This table illustrates that while need gradually causes propensity to work in public service to increase, taking out additional debt lowers that probability at a much greater rate, to the point of being vanishingly small for students with greater than \$15,000 worth of loans.

	\begin{table}
		\centering
		\caption{Second stage results on major choice}	
		\resizebox{\textwidth}{!}{	
			\input{\regs majChoice.tex}
		}
		\label{majChoice}
	\end{table}

	\begin{table}
		\centering
		\caption{Second stage marginal effects on major choice given extra \$1,000 of debt}	
		\resizebox{\textwidth}{!}{	
			\input{\regs majChoicemarg.tex}
		}
		\label{majChoicemarg}
	\end{table}

	\begin{table}
		\centering 
		\caption{Probability of working in public service for various financial backgrounds}
		\input{\regs simResults.tex}
		\label{probest}
	\end{table}
	
	It is worth noting however that unlike Rothstein and Rouse, I find no impact on wages, as shown in Table \ref{incomeRes}. The marginal effects on the threshold probits are visible in Table \ref{incomeResmarg}. Regression (1) here performs a 2SLS regression examining the income of students two years after graduation. To test for differential effects at different income levels, I also run probits in regressions (2) and (3) to test whether loans can push people over certain thresholds, defined to be the 25th and 90th percentiles of income in my sample. While Rothstein and Rouse find that more debt leads students to avoid uniquely low paying positions, I do not replicate this result. While it is possible these mismatches are simply a failure in statistical power, it is also possible that this is a difference between Rothstein and Rouse's wealthier sample and my lower-income one. 
	
	One explanation of this distinction could be differences in flexibility within a career path between classes. If students have varying levels of foresight, some students may make career choices in the context of their student debt early in their education while others could be more myopic. In this case, those with greater foresight would anticipate difficulty paying back student loans in public service careers, and thus avoid those careers, leading to the reduction seen in both my sample and Rothstein and Rouse's. 
	
	However, suppose not all students anticipate this difficulty--\textcite{smith2013} suggest many students are overconfident in their ability to repay their student debt. If these overconfident students ignore their risk of default, it is possible they would choose $L$ when they would really prefer the security provided by $H$'s higher salary. This is where it is possible my sample differs from Rothstein and Rouse's. Given that upper class students are more able to pivot into new careers after graduation, as suggested by \textcite{mcleod2009}, then Anon U's public service students may have been able to switch tracks into a higher paying job while the lower-income NLSY students could not. In this case, that would explain why those with debt at Anon U would be able raise their salaries after graduation while the indebted in my sample could not.
	
	\begin{table}
		\centering
		\caption{2SLS Regression Predicting Income}
		\resizebox{\textwidth}{!}{		
			\input{\regs income.tex}
		}
		\label{incomeRes}
	\end{table}

	\begin{table}
		\centering
		\caption{2SLS Marginal Effects on Income Thresholds}
		\resizebox{\textwidth}{!}{		
			\input{\regs incomemarg.tex}
		}
		\label{incomeResmarg}
	\end{table}
	
	\section{Discussion and Conclusions}
	
	Contrary to the existing literature, I find no relationship between student debt and income two years post graduation. This may suggest that low-income students are less able to recover from overconfidence in their borrowing choices than higher-income students. Reinforcing findings from other studies, I do find that students with debt are significantly less likely to enter public service industries such as teaching, with the rate declining roughly 5 percent per extra thousand dollars of debt. Given these are occupations which frequently provide positive externalities to their communities as shown in \textcite{benshem1991}, I would describe this as a negative byproduct of higher education's reliance on student loans for education financing.
	
	However, to advocate against borrowing for education on these grounds without considering access is an incomplete argument. As seen in \textcite{abel2014}, college is increasingly necessary for a middle class existence in the United States. Thus, if we hope to both maintain an educated workforce and allow for social mobility, we must find a better solution which subjects students to less risk regarding loan default. While loan subsidies reduce the cost of a loan, therefore reducing the number of payments required and thus the probability of default, it does so for all borrowers, not just those who would struggle to repay. Further, interest rate subsidies have little impact early in the borrower's life cycle, as they must make a comparable payment whether the interest is subsidized or not. The benefit of interest subsidies comes later in life, as borrowers can finish repaying their debts earlier than they otherwise would. According to \textcite{dynarski2015}, this system is backwards--borrowers receive little benefit early out of school when their earnings are lowest, but significant relief later in life when their earnings are at their peak. Given that students are most at risk of default when their incomes are low, these interest subsidies fail to help students avoid default when they are most in need, and so are a poorly targeted tool for helping students repay their loans. 
	
	\textcite{abraham2018} suggest using income based repayment (or IBR) as a means of mitigating the risks of student loan default. Returning to the model from section 3, IBR raises the cost of borrowing for those on career path $H$, as they must pay more of their income, while lowering the cost for those on path $L$ since the risk of default is significantly reduced. Both lowering the probability of default and lowering the expected earnings of $H$ push more students towards career path $L$. If, as suggested by \textcite{benshem1991} and \textcite{hanson1995}, careers with high $C$ are more likely to be altruism-focused, then IBR could significantly boost positive externalities targeted towards those in need.  Oregon and Michigan's ``Pay It Forward'' programs, both implemented in 2014 are already exploring these IBR programs at a state level, with the first graduates starting to emerge using these repayment plans. However, this is also not a perfect scheme. Similar to any insurance market, IBR is a target for adverse selection. If only students expecting remarkably low earnings take on IBR loans, then the lender will very rarely be repaid the full amount as debt for low earners is traditionally forgiven over time \parencite{dynarski2015}. One solution to this problem would be to find a way for lenders to pool high- and low-earners so that high earners repaying more than their loan can balance out the failed payments of low earners.
	
	Beyond the claims made in  \textcite{dynarski2015} that these subsidies fail to encourage college attendance, I show that by burdening students with more debt we pressure them out of pro-social positions such as teaching. Given that these sectors are frequently chosen by altruistic workers, it is easy to believe they are more likely to provide positive externalities to their community. Thus, I find suggestive evidence that interest rate subsidies for college loans are detrimental to societal well-being as they push students away from career paths which are socially beneficial, leading to a societal loss of many positive externalities from these pro-social careers.
	
	
	
	
	\clearpage
	\appendix
	\section{Extended Regressions}
	
	\begin{table}
		\centering
		\caption{Naive regression (table \ref{naive2}) full results}
		\resizebox{\textwidth}{!}{
			\input{\regs naivefull.tex}
		}
		\label{naivefull}
	\end{table}

	\begin{table}
		\centering
		\caption{Naive regression marginal effects (table \ref{naive2marg}) full results}
		\resizebox{\textwidth}{!}{
			\input{\regs naivefullmarg.tex}
		}
	\end{table}

	\begin{table}
		\centering
		\tiny
		\caption{First stage (table \ref{firststage}) full results}
		\input{\regs tripdiffull.tex}
		\label{firststagefull}
	\end{table}

	\begin{table}
		\centering
		\tiny
		\caption{Major selection (table \ref{majChoice}) full results}
		\input{\regs majChoicefull.tex}
		\label{majChoiceFull}
	\end{table}

	\begin{table}
		\centering
		\scriptsize
		\caption{Major selection marginal effects(table \ref{majChoicemarg}) full results}
		\input{\regs majChoicefullmarg.tex}
		\label{majChoiceFullmarg}
	\end{table}

	\begin{table}
		\centering
		\tiny
		\caption{Income (table \ref{incomeRes}) full results}
		\input{\regs incomefull.tex}
	\end{table}

	\begin{table}
		\centering
		\scriptsize
		\caption{Income marginal effects (table \ref{incomeResmarg}) full results}
		\input{\regs incomefullmarg.tex}
	\end{table}
	\clearpage
	
	\printbibliography
	
\end{document}